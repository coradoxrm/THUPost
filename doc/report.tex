%!TEX program = xelatex
\documentclass[11pt, a4paper]{article}

\usepackage{amsmath}
\usepackage{listings}
\usepackage{amssymb}

% fonts
\usepackage{xeCJK}
\setCJKmainfont[BoldFont=SimHei]{SimSun}
\setCJKfamilyfont{hei}{SimHei}
\setCJKfamilyfont{kai}{KaiTi}
\setCJKfamilyfont{fang}{FangSong}
\newcommand{\hei}{\CJKfamily{hei}}
\newcommand{\kai}{\CJKfamily{kai}}
\newcommand{\fang}{\CJKfamily{fang}}

% style
\usepackage[top=2.54cm, bottom=2.54cm, left=3.18cm, right=3.18cm]{geometry}
\linespread{1.5}
\usepackage{indentfirst}
\parindent 2em
\punctstyle{quanjiao}
\renewcommand{\today}{\number\year 年 \number\month 月 \number\day 日}

% figures and tables
\usepackage{graphicx}
\usepackage[font={bf, footnotesize}, textfont=md]{caption}
\makeatletter
    \newcommand\fcaption{\def\@captype{figure}\caption}
    \newcommand\tcaption{\def\@captype{table}\caption}
\makeatother
\usepackage{booktabs}
\renewcommand\figurename{图}
\renewcommand\tablename{表}
\newcommand{\fref}[1]{\textbf{图 \ref{#1}}}
\newcommand{\tref}[1]{\textbf{表 \ref{#1}}}
\newcommand{\tabincell}[2]{\begin{tabular}{@{}#1@{}}#2\end{tabular}} % multiply lines in one grid
\usepackage{longtable} % long table

\usepackage{listings}
\lstset{basicstyle=\ttfamily}

\usepackage{xcolor}
\renewcommand{\r}{\color{red}}
\usepackage{tabulary}
% start of document
\title{\textbf{技术文档}}
\author{
    \kai 钱迪晨 \quad 计35 \quad 2013011402 \\
    \kai 任淼 \quad 计35 \quad 2013011394 \\
    \kai 朱俸民 \quad 计35 \quad 2012011894
}
\date{\kai\today}

% -----------------start here------------------%
\begin{document}
% \lstset{                        %Settings for listings package.
%   language=[ANSI]{C++},
%   % backgroundcolor=\color{lightgray},
%   basicstyle=\footnotesize,
%   breakatwhitespace=false,
%   breaklines=true,
%   captionpos=b,
%   commentstyle=\color{olive},
%   directivestyle=\color{blue},
%   extendedchars=false,
%   % frame=single,%shadowbox
%   framerule=0pt,
%   keywordstyle=\color{blue}\bfseries,
%   morekeywords={*,define,*,include...},
%   numbersep=5pt,
%   rulesepcolor=\color{red!20!green!20!blue!20},
%   showspaces=false,
%   showstringspaces=false,
%   showtabs=false,
%   stepnumber=2,
%   stringstyle=\color{purple},
%   tabsize=4,
%   title=\lstname
% }

\maketitle

\section{框架}
本次大作业我们使用的是ruby on rails框架,部署使用的是nginx+unicorn,部署的机器是azure,已经购买了thupost.cn的域名。

具体在web端我们使用了如下技术。

\begin{enumerate}
    \item Ruby on Rails 2.3.1.
    \item Bootstrap
    \item MySQL
    \item jQuery
\end{enumerate}


\section{使用的服务}
我们使用了阿里大云\url{http://www.alidayu.com}提供的短信服务。

对于email我们在服务器上自己搭建了smtp服务器,用来进行邮件提醒的发送功能。

我们使用了清华邮箱进行用户身份验证,保障只有清华的学生可以使用THUPOST。
\section{API列表}

\begin{center}
    \tcaption{order API列表}\label{table:order_api_list}
    \begin{longtable}{lll}
        \toprule
        名称 & action & 描述\\
        \midrule
        order & create & 新建一个订单\\
        order & index & 用户所有订单\\
        order & notify\_email & email提醒 \\
        order & notify\_text & 短信提醒 \\
        order & remove & 删除订单 \\
        \bottomrule
    \end{longtable}
\end{center}

\begin{center}
    \tcaption{product API列表}\label{table:product_api_list}
    \begin{longtable}{lll}
        \toprule
        名称 & action & 描述\\
        \midrule
        product & show & 展示一个物品\\
        product & create & 新建一个物品\\
        product & for\_sale & 用户出售的物品 \\
        product & search & 查询页面 \\
        product & tag & 根据tag页面 \\
        product & remove & 删除一个物品 \\
        product & selled & 标记一个物品已经卖出 \\
        product & cancel & 标记一个正在联系的物品取消联系 \\
        \bottomrule
    \end{longtable}
\end{center}

\begin{center}
    \tcaption{user API列表}\label{table:user_api_list}
    \begin{longtable}{lll}
        \toprule
        名称 & action & 描述\\
        \midrule
        users & sign\_in & 登录\\
        users & sign\_out & 登出\\
        users & sign\_up & 注册\\
        users & password/new & 重置密码 \\
        users & password/edit & 修改密码 \\
        users & confirmation & 确认激活 \\
        users & confirmation/new & 重发激活邮件 \\
        user\_edit & show & 个人资料 \\
        user\_edit & edit & 更新个人资料 \\
        \bottomrule
    \end{longtable}
\end{center}


% \begin{split}
% S &= \frac{1}{(1-0.5) + \frac{0.5}{10}}\\
% S &= \frac{1}{0.5 + 0.05} \\
% S &= 1.818
% \end{split}
% \]
% \begin{center}
%     \includegraphics[height=10cm]{result/9_epoch_1.jpg}
%     \fcaption{第1轮迭代训练的结果}\label{1}
% \end{center}

% \begin{enumerate}
%     \item 随着实验的进行,图片的确变得更加的准确,而不仅仅是肉眼上的复原效果更加好了,但是训练的次数过多也会使得画质变得更差。
%     \item 10次迭代的时候,答案是最优的。
%     \item 绿色通道随着迭代次数上升最准确,但是红色和蓝色通道就有一点不正确。
% \end{enumerate}


% \begin{center}
%     \tcaption{内存地址空间映射}\label{table:mem_addr}
%     \begin{longtable}{ll}
%         \toprule
%         逻辑地址 & 映射到的物理地址或设备 \\
%         \midrule
%         0x0000 $\thicksim$ 0x7FFF & RAM2 0x0000 $\thicksim$ 0x7FFF \\
%         0x8000 $\thicksim$ 0xBEFF & RAM1 0x0000 $\thicksim$ 0x3EFF \\
%         \bottomrule
%     \end{longtable}
% \end{center}

\end{document}
% -----------------end------------------%
